% !TeX spellcheck = en_GB
\section*{Abstract}

The field of machine learning (AutoML) has been highlighted as one of the main alternatives for finding good solutions for complex machine learning problems. Despite the recent success of AutoML, many challenges remain. Learning AutoML is a time-consuming process and can be computationally inefficient. Meta-learning is described as the process of learning from past experiences by applying various learning algorithms on different types of data, and therefore reduces the time required to learn new tasks. One of the advantages of meta-learning techniques is that they can serve as an efficient support for the AutoML process, learning from previous tasks the best algorithms to solve a certain type of problem. In this way, it is possible to speed up the AutoML process, obtaining better results in the same period of time. The objective of this thesis is to design a meta-learning strategy for generic domains in machine learning.

The implemented meta-learning proposal is capable of tackling a wide variety of tasks by selecting features capable of representing the space defined by them.
% In addition, by exploring the interaction between the characteristics of the datasets and the structure of the flows, the proposed method is capable of identifying flows with good performance without performing a computationally expensive analysis.
As a complementary AutoML system, AutoGOAL was chosen, which stands out for its ability to generate effective solutions for a wide range of domains, allowing it to solve many tasks. AutoGOAL is used for the generation of machine learning pipelines to create the knowledge base and in the search for pipelines initialized with the designed meta-learning approach. % en Due to the variety of machine learning tools it uses, there is a great diversity in the saved flows.

The meta-learning approach developed consists of selecting a set of machine learning pipelines to be proposed in the initialization of the optimization process of an AutoML system. The choice of this set of pipelines is carried out using a ranking approach, in which the \texttt{k} best piplines are selected for a new dataset. For this, several strategies were implemented. The experimental evaluation carried out on a large number of datasets shows that these meta-learning strategies obtain better results in terms of the pipelines found with respect to AutoGOAL without meta-learning.

%
%\begin{center}
%	{\large \textbf{Objetivos de la tesis}}
%\end{center}
%\qquad 
%
%\qquad \textbf{Objetivo general de la tesis}: 
%
%\qquad 
%
%\qquad Desarrollo de codig\'{o}s adaptativos para la integraci\'{o}n de
%problemas de valor inicial de dimensiones no peque\~{n}as basados en
%esquemas embebidos de linealización local de orden superior. 
%
%\qquad 
%
%\qquad \textbf{Objetivos espec\'{\i}ficos de la tesis:}
%
%\qquad 
%
%\qquad 1-) Construcción de nuevas f\'{o}rmulas embebidas de Dormand y Prince
%localmente linealizadas para problemas de valor inicial de dimensiones no
%peque\~{n}as.
%
%\qquad 
%
%\qquad 2-) Construcción de c\'{o}digos adaptativos para la selecci\'{o}n
%autom\'{a}tica del tama\~{n}o de paso y de la dimensión de los subespacios
%de Krylov en las nuevas f\'{o}rmulas embebidas basados en nuevas estrategias
%de selecci\'{o}n de la dimensión de los subespacios de Krylov, para el
%control del BreakDown en el algoritmo de ortogonalizaci\'{o}n de Arnoldi, y
%en la posible reutilizaci\'{o}n del Jacobiano evaluado en pasos de integraci\'{o}n anteriores.
%
%\qquad 
%
%\qquad 3-) Evaluación de las potencialidades de las nuevas fórmulas embebidas y códigos adaptativos mediante simulaciones numéricas.
%
%\qquad 