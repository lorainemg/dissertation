\begin{opinion}


En la actualidad el Aprendizaje Automático ha llegado a todas las ramas de la industria, ayudando a resolver un gran número de problemas pero 
creando la necesidad de un enorme número de expertos para poder utilizar las herramientas adecuadas en cada caso.
En este escenario el AutoML propone una solución ayudando con la selección de forma automática de las mejores soluciones con el problema añadido de que incrementa
el costo computacional ya que tiene que evaluar muchas soluciones para resolver cada problema. %Realizando esta tarea cada vez.
El área de investigación en que incursiona la estudiante propone un enfoque para que los sistemas de AutoML puedan aprender de su experiencia y 
disminuir el tiempo de respuesta y aumentar la calidad de las soluciones encontradas.

La estudiante Loraine Monteagudo en esta investigación se adentró en un tema del estado del arte de gran actualidad y para eso tuvo que utilizar 
conocimientos de varias asignaturas de la carrera y otros que no son parte del curriculum estandar.
Su propuesta implicó el diseño de características para describir problemas de aprendizaje de máquina, el diseño e implementación de varias estrategias de meta-learning y 
una amplia experimentación con cientos de datasets del estado del arte.
Además implicó conocer una herramienta de AutoML nueva e incorporarle su estrategia para evaluar y comparar sus resultados en la práctica.

Sus resultados resultan muy prometedores, reportando mejoras que reducen a la mitad el error final de la solución encontrada, utilizando el mismo costo computacional.
Esta mejora es considerable para una herramienta del estado del arte, que ya lograba resultados comparables a las mejores herramientas de AutoML existentes.
Más aún, las estrategias desarrolladas en esta investigación han sido aplicadas solamente a una parte pequeña del proceso de 
AutoML --la inicialización de la población de soluciones-- pero pueden ser extendidos fácilmente a todo el proceso de optimización.

Loraine ha formado parte del grupo de investigación de inteligencia artificial desde sus inicios en la carrera.
Desde entonces ha participado en cuanto jornada científica, taller, y fórum se le ha puesto delante, siempre con una buena dosis de creatividad y una gran cantidad de esfuerzo.
Esta tesis no es menos, presenta resultados de primer nivel que son susceptibles de ser publicados en los mejores eventos científicos del área del AutoML, y demuestra con creces que Loraine ha alcanzado las habilidades y la madurez científica que esperamos de los mejores graduados de Ciencia de la Computación.
Con esta tesis, Loraine cierra una de las etapas más importantes de su formación profesional, y se convierte así, por derecho propio, en una miembro más de este gremio de científicos, donde seguro tendrá una carrera tan brillante como ha sido hasta ahora su vida de estudiante.


\begin{center}
	\begin{tikzpicture}[x=.01\linewidth, y=.01\linewidth]
	\node[anchor=south west,inner sep=0] (image) at (6, -4) {\includegraphics[width=.21\linewidth]{Graphics/suilan.png}};
	\draw[-, thick] (-1, 0) edge (31, 0);
	\node[below,align=center,text width=.4\linewidth] at (15, -1) {Msc. Suilan Estévez Velarde};
	\end{tikzpicture}
	\begin{tikzpicture}[x=.01\linewidth, y=.01\linewidth]
	\node[anchor=south west,inner sep=0] (image) at (6, -3) {\includegraphics[width=.21\linewidth]{Graphics/daniel.jpg}};
	\draw[-, thick] (-1, 0) edge (31, 0);
	\node[below,align=center,text width=.4\linewidth] at (15, -1) {Lic. Daniel Alejandro Valdés Pérez};
	\end{tikzpicture}
%	\begin{tikzpicture}[x=.01\linewidth, y=.01\linewidth]
%	%\node[anchor=south west,inner sep=0] (image) at (0, 0) {\includegraphics[width=.3\linewidth]{}};
%	\draw[-, thick] (-1, 0) edge (31, 0);
%	\node[below,align=center,text width=.3\linewidth] at (15, -1) {Dra. Marta Lourdes Baguer D\'iaz-Roma\~nach};
%	\end{tikzpicture}
\end{center}

\end{opinion}