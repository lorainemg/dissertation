\begin{abstract}

\end{abstract}
\newpage
%
%\begin{center}
%	{\large \textbf{Objetivos de la tesis}}
%\end{center}
%\qquad 
%
%\qquad \textbf{Objetivo general de la tesis}: 
%
%\qquad 
%
%\qquad Desarrollo de codig\'{o}s adaptativos para la integraci\'{o}n de
%problemas de valor inicial de dimensiones no peque\~{n}as basados en
%esquemas embebidos de linealización local de orden superior. 
%
%\qquad 
%
%\qquad \textbf{Objetivos espec\'{\i}ficos de la tesis:}
%
%\qquad 
%
%\qquad 1-) Construcción de nuevas f\'{o}rmulas embebidas de Dormand y Prince
%localmente linealizadas para problemas de valor inicial de dimensiones no
%peque\~{n}as.
%
%\qquad 
%
%\qquad 2-) Construcción de c\'{o}digos adaptativos para la selecci\'{o}n
%autom\'{a}tica del tama\~{n}o de paso y de la dimensión de los subespacios
%de Krylov en las nuevas f\'{o}rmulas embebidas basados en nuevas estrategias
%de selecci\'{o}n de la dimensión de los subespacios de Krylov, para el
%control del BreakDown en el algoritmo de ortogonalizaci\'{o}n de Arnoldi, y
%en la posible reutilizaci\'{o}n del Jacobiano evaluado en pasos de integraci\'{o}n anteriores.
%
%\qquad 
%
%\qquad 3-) Evaluación de las potencialidades de las nuevas fórmulas embebidas y códigos adaptativos mediante simulaciones numéricas.
%
%\qquad 